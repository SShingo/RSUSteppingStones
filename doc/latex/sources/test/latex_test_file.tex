\documentclass[a4j, 10pt]{ltjsbook}
\def\RSULocale{ja_jp}
\def\DocStrNoReturnjajp{戻り値なし}
\def\DocStrNoReturnenus{No Return Value}
%上下幅設定
\setlength{\textheight}{\paperheight}   % 本文領域を一旦紙面の高さにする
\setlength{\topmargin}{13.6truemm}      % 上の余白を39mm(=1inch+13.6mm)に
\addtolength{\topmargin}{-\headheight}  % 
\addtolength{\topmargin}{-\headsep}     % ヘッダの分だけ上の余白を小さくする
\addtolength{\textheight}{-74truemm}    % 下の余白35mmと合わせて、上下余白分だけ本文領域の高さを小さくする。

%左右幅設定
\setlength{\fullwidth}{\paperwidth}     % 全体の幅(テキスト領域の幅ではない)を一旦紙面幅にする
\setlength{\evensidemargin}{-0.4truemm} % 左の余白を25mm(=1inch-0.4mm)に
\setlength{\oddsidemargin}{-0.4truemm}  % 上に同じ
\addtolength{\fullwidth}{-50truemm}     % 左右余白分だけ本文領域の幅を小さくする。
\setlength{\textwidth}{\fullwidth}      % テキスト幅を全体の幅に一致させる
\setlength{\tabcolsep}{6pt} 

\usepackage{amsmath, amsfonts, amssymb, bm}
\usepackage{framed}
\usepackage{ntheorem}
\usepackage{graphicx, color}
\usepackage[many]{tcolorbox}
\usepackage{longtable}
\usepackage{tabu}
\usepackage{listings,jvlisting}
\usepackage[Bjornstrup]{fncychap}
\usepackage{makecell}
\usepackage{xltabular}
\usepackage{pifont}
\usepackage{fancybox}
%%%プリアンブルに下記を書く
\newtcolorbox{marker}[1][]{enhanced,
  before skip=2mm,after skip=3mm,fontupper=\gtfamily\sffamily,
  boxrule=0.4pt,left=5mm,right=2mm,top=1mm,bottom=1mm,
  colback=yellow!50,
  colframe=yellow!20!black,
  sharp corners,rounded corners=southeast,arc is angular,arc=3mm,
  underlay={%
    \path[fill=tcbcolback!80!black] ([yshift=3mm]interior.south east)--++(-0.4,-0.1)--++(0.1,-0.2);
    \path[draw=tcbcolframe,shorten <=-0.05mm,shorten >=-0.05mm] ([yshift=3mm]interior.south east)--++(-0.4,-0.1)--++(0.1,-0.2);
    \path[fill=yellow!50!black,draw=none] (interior.south west) rectangle node[white]{\Huge\bfseries !} ([xshift=4mm]interior.north west);
    },
  drop fuzzy shadow,#1}
%%%プリアンブルに上記を書く
\definecolor{problemblue}{RGB}{100,134,158}
\definecolor{idiomsgreen}{RGB}{0,162,0}
\definecolor{exercisebgblue}{RGB}{192,232,252}

\newtcolorbox{DefFunc}[2][]{
  breakable,
  enhanced,
  colback=white,
  colframe=exercisebgblue,
  arc=0pt,
  outer arc=0pt,
  title=#2,
  fonttitle=\bfseries\tt\large,
  colbacktitle=exercisebgblue,
  coltitle = black,
  attach boxed title to top left={},
  boxed title style={
    enhanced,
    skin=enhancedfirst jigsaw,
    arc=3pt,
    bottom=0pt,
    interior style={fill=exercisebgblue}
  }
}

\begin{document}
\chapter*{お断り}
\begin{itemize}
\item 本文書、およびRSU Development Moduleの著作権は著作者に属します。
\item RSU Development Moduleの仕様、および本文書の内容は将来予告なしに変更することがあります。
\item 本文書で使用されている製品名、社名などは、一般にその所有者の商標または登録商法です。
\item 著者は本文書、およびRSU Development Moduleの完全性・正確性を保障致しません。いかなる場合においても、著者は本文書、およびRSU Development Moduleに関連して生じた通常の直接的、間接的、必然的、偶発的、特別な、あるいは懲罰的賠償について、たとえ著者がそのような賠償が発生する可能性があることを通告されたとしても、何ら責任を負いません。
\end{itemize}
\vskip\baselineskip
\noindent【著作者】\par
\noindent 鈴木慎吾\par
\noindent SAS Institute Japan株式会社\par
\noindent Risk Solution統括部\par
\noindent\texttt{shingo.suzuki@sas.com}
\vskip2\baselineskip
\begin{center}
\doublebox{%
\begin{minipage}{12truecm}
本書の内容の全部または一部を無断で複写,複製,転記載および磁気または光記録媒体等への入力を行うことは,法律で定められた場合を除き,著作権の侵害となりますので,その場合はあらかじめ小社あて許諾を求めて下さい.
\end{minipage}}
\end{center}
これはテストです\DocStrNoReturnjajp さて。、、ここでテキスト領域の幅が幾らにせっていされているかを確認しておこう。
\begin{center}
\begin{tabularx}{\textwidth}{|p{10\zw}|X|l|}
	\hline
	\makecell[c]{鈴木慎吾慎吾}&suzuki&\ding{52}machiko
\end{tabularx}
\end{center}

\begin{marker}
これらグローバル変数の値は書き換え可能ですが(モジュール側で書き換え制限を課せられないため)、絶対にプログラム中に削除したり、書き換えたりしないでください。予測不能な結果をもたらす可能性があります。
\end{marker}

\begin{marker}
	モジュール本体のデータセットの中身は絶対に変更しないでください。不用意にデータセットを改変すると起動不可能になります。
\end{marker}

\chapter{xxx}
\section{yyy}
\subsection{zzz}
\begin{center}
	{\footnotesize
	\begin{xltabular}{\textwidth}{|p{5truecm}|X|p{5truecm}|}
	\hline
	\thead{DocStrConstantName}&\thead{DocStrConstantDescriptopm}&\thead{DocStrConstantValue}\\
	\hline
	\hline
	\texttt{RSUArray}&配列パッケージ Prefix&\texttt{RSUArray\_\_}\\
	\hline
	\texttt{RSU\_G\_CLASS\_FILE\_ITERATOR}&イテレータクラス定義ファイル名&\texttt{RSU\_PKG\_Class\_IteratorArray}\\
	\hline
	\texttt{RSUClass}&クラスパッケージ Prefix&\texttt{RSUClass\_\_}\\
	\hline
	\texttt{RSUCounter}&配列パッケージ Prefix&\texttt{RSUCounter\_\_}\\
	\hline
	\texttt{RSU\_G\_CLASS\_FILE\_COUNTER}&カウンタークラス定義ファイル名&\texttt{RSU\_PKG\_Class\_Counter}\\
	\hline
	\texttt{RSU\_G\_CLASS\_FILE\_PROGRESS\_BAR}&プログレスバークラス定義ファイル&\texttt{RSU\_PKG\_Class\_ProgressBar}\\
	\hline
	\texttt{RSUDate}&日付パッケージ Prefix&\texttt{RSUDate\_\_}\\
	\hline
	\texttt{RSUDebug}&デバッグパッケージ Prefix&\texttt{RSUDebug\_\_}\\
	\hline
	\texttt{RSUDic}&連想配列パッケージ Prefix&\texttt{RSUDic\_\_}\\
	\hline
	\texttt{RSUDir}&ディレクトリパッケージ Prefix&\texttt{RSUDir\_\_}\\
	\hline
	\texttt{RSUDS}&データセットパッケージ Prefix&\texttt{RSUDS\_\_}\\
	\hline
	\texttt{RSU\_G\_CLASS\_FILE\_DS\_ITERATOR}&データセットイテレータクラス定義ファイル&\texttt{RSU\_PKG\_Class\_IteratorDS}\\
	\hline
	\texttt{RSUExcel}&エクセルファイルパッケージ Prefix&\texttt{RSUExcel\_\_}\\
	\hline
	\texttt{RSUFile}&ディレクトリパッケージ Prefix&\texttt{RSUFile\_\_}\\
	\hline
	\texttt{RSU\_G\_CLASS\_FILE\_FILEWRITER}&ファイルライタークラス定義ファイル名&\texttt{RSU\_PKG\_Class\_FileWriter}\\
	\hline
	\texttt{RSU\_G\_CLASS\_FILE\_FILE\_ITERATOR}&ファイルイテレータクラス定義ファイル名&\texttt{RSU\_PKG\_Class\_IteratorFile}\\
	\hline
	\texttt{RSU\_G\_ARRAY\_DELIMITER}&配列要素の区切り文字&\texttt{`}\\
	\hline
	\texttt{RSU\_G\_INSTANCE\_PREFIX}&インスタンス名に付与されるprefix&\texttt{RI\_}\\
	\hline
	\texttt{RSU\_G\_INSTANCE\_ID\_DIGIT}&インスタンス名のID部の桁数&\texttt{4}\\
	\hline
	\texttt{RSU\_G\_DATASET\_ID\_DIGIT}&データベース名のID部の桁数&\texttt{6}\\
	\hline
	\texttt{RSU\_G\_MACRO\_DEF\_TMP\_FILE}&クラス定義用一時ファイル名&\texttt{rsu\_class\_instantiate}\\
	\hline
	\end{xltabular}
	}
	\end{center}
	//
	//

	{\small
	\begin{DefFunc}{ContainsItem}
		指定要素が配列に含まれているか否かを返します。\\\\
		\begin{tabular}{rl}
			\makecell[r]{\bfseries 定義:}&\begin{minipage}[t]{12truecm}\begin{verbatim}
%macro RSUArray__ContainsItem(
          ivar_array
          , i_item
       );
\end{verbatim}\end{minipage}\\\\
			\makecell[r]{\bfseries 戻り値:}&0: 指定要素なし\quad 1: 指定要素あり\\\\
			\makecell[r]{\bfseries 引数:}&\begin{minipage}[t]{15truecm}\vspace*{-7pt}\begin{tabularx}{13.5truecm}{|l|X|c|}
\hline
\thead{変数名}&\thead{説明}&\thead{必須}\\
\hline
\hline
\texttt{ivar\_array}&配列を保持しているマクロ変数名&\ding{51}\\
\hline
\texttt{i\_item}&検索要素&-\\
\hline
\end{tabularx}\end{minipage}\\
		\end{tabular}
	\end{DefFunc}
	}

	\begin{verbatim}
		%macro test(i_test)
	\end{verbatim}
	
		\end{document}